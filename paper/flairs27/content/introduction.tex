
\section{Introduction}

Wikipedia.org (WP) is the largest and most popular general reference work on the Internet.
The website is estimated to have nearly 365 million readers worldwide.
An important part of keeping WP usable it to include new and current content.
%An important challenge in maintaining WP, the most popular web-based, collaborative, multilingual KB on the internet, is making sure its contents are up-to-date.
Presently, there is considerable time lag between the publication of an event and its citation in WP\@.
The median time lag for a sample of about 60K web pages cited by WP articles in the \textit{living\_people} category is over a year and the distribution has a long and heavy tail~\cite{JFrank12}.
%Also, the majority of WP entities have updates on their associated article much less frequently than their mention frequency.
Such stale entries are the norm in any large reference work because the number
of humans maintaining the reference is far fewer than the number of entities.

Reducing latency keeps WP relevant and helpful to its users.
Given an entity page, such as \textsl{wiki/Boris\_Berezovsky\_(businessman)}\footnote{http://en.wikipedia.org/wiki/Boris\_Berezovsky\_(businessman)},
possible citations may come from a variety of sources.
Notable news may be derived from newspapers, tweets, blogs and a variety of
different sources include Twitter, Facebook, iBlogs, arxiv, etc.
However, the actual citable information is a small percentage of the total documents that appear on the web.
%We develop a system to read streaming data and filter out articles that are candidates for citations. 
%The goal of this system is to read web documents and recommend citable facts (attributes) for WP pages.
To help WP editors, a system is needed to parse through terabytes of documents 
and select facts that can be recommended to particular WP pages.


Previous approaches are able to find relevant documents given a list of WP
entities as query nodes \cite{mcnamee2012hltcoe, dalton2013bi,
Bonnefoy:2013:WDE:2484028.2484180, Balog:2013:CCR:2484028.2484151,ji2011knowledge}.
Entities of three categories \textit{person}, \textit{organization} and \textit{facility} are considered.
This work involves processing large sets of information to determine which facts may contain references to a WP entity. 
This problem becomes increasingly more difficult when we look to extract relevant facts from
each document.
Each relevant document must now be parsed and processed to determine if a sentence or paragraph is worth being cited.
% Reorder this paragraph
%The challenges that we address are: first finding documents that contain useful information about the entity (avoid spams, documents that do not have direct information about the entity even though they mention them, etc), the scale where \textit{stream processing} nature of the system makes it suitable to avoid batch processing natures of Hadoop and operate in the realm of streams such as twitter storm~\footnote{http://storm-project.net/} which similarly processes unblounded streams of twitter data or Spark Streaming~\footnote{http://spark.incubator.apache.org/}. Further, this is called streaming because data is being generated as time goes
%on and for each extraction we should only consider current or past data. As other Natural Language Processing tasks, precision and recall of the extent we can extract slot values are very important metrics that we will discuss later on. Having to balance between infamous and obscure entities, dealing with slots that can be very broad (e.g.\ things that an entity is affiliated with) or very specific (such as cause of death of a \textit{person} entity).

Discovering facts across the Internet that are relevant and citable to the WP entities is a non-trivial task.
%For example, a take a sentence from the Internet:
Here we produce an example sentence from a webpage: 

%\texttt{Boris Berezovsky made his fortune in Russia in the 1990s when the
%country went through privatisation of state property and `robber capitalism', and passed away March 2013.}
``{\small \texttt{Boris Berezovsky, who made his fortune in Russia in the 1990s, passed away March 2013.}}''

After parsing the sentence, we must first note that there are two entities named \textit{Boris Berezovsky} WP; one a businessman and the other a pianist.
Any extraction needs to take this into account and employ a viable distinguishing policy (entity resolution).
Then, we match the sentence to find a topic such as \textit{DateOfDeath} valued at \textit{March 2013}.
Each of these operations is expensive so an efficient framework is necessary to execute these operations at web scale.
%Other examples that this system can answer could be `Who a person has met during a certain period of time?', `Who are the employees of this organization?', `Who has met who in this facility at a certain time?'.


In this paper, we introduce an efficient fact extraction system or given WP entities from a time-ordered document stream.
Fact extraction is defined as follows: match each sentence to the generic sentence structure of \{\textit{subject} --- \textit{verb} --- \textit{adverbial/complement}\}~\cite{sentencePatterns08}.
The first \textit{subject} represents the entity (WP entity) and \textit{verb} is the relation type (slot) we are interested in (e.g. Table~\ref{table:slotNameOntology}).
The third component, \textit{adverbial/complement}, represents the value of the associated slot.
In our example sentence, the entity of the sentence is \textit{Boris Berezovsky} and the slot we extract is
\textit{DateOfDeath} with a slot value of \text{March 2013}.
The resulting extraction containing an entity, slot name and slot value is a \emph{fact}.


%Slot extraction is a challenging task in current state of the art Knowledge Bases (KB).
%Popular graphical KB such as Freebase or DBPedia keep data in structured format where entities
%are connected via relationships (slots) and the associated attributes
%(slot values).
%Our system can be used as an aid to recommend factual updates to such KBs.


Our system contains three main components.
First, we pre-process the data and build models representing the WP query entities.
Next, we use the models to filter a large stream of documents so they only contain candidate citations.
Lastly, we processes sentences from candidate extractions and return slot values. 
% EITHER SAY WHAT IT MEANS OR GIVE EXAMPLES OR COMMENT IT
%We build a modular system that allows us to explore the nuances of the training data and queries. 
Overall, we contribute the following:
\begin{itemize}[noitemsep,nolistsep]
\item Introduce a method to build models of WP name variations;% (Section~\ref{sec:entitymodel});
\item Built a system to filter a large amount of diverse documents using a natural language processing rule-based extraction system;% (Section~\ref{sec:slotfilling});
\item Extract, infer and filter entity-slot-value triples of information to be added to KB.%(Section~\ref{sec:constraintsandinference}).
\end{itemize}

\begin{table}
\caption{The set of possible slot name for each entity type.}
\centering
\label{table:slotNameOntology}

\setlength{\tabcolsep}{1pt}

\begin{tabular}{|c|c|c|}
\hline 
\textbf{{\small Person}} & \textbf{{\small Facility}} & \textbf{{\small Organization}} \\ 
\hline 
\begin{tabular}{@{}l@{}}{\small Affiliate }\\ {\small AssociateOf }\\  {\small Contact\_Meet\_PlaceTime} \\ {\small AwardsWon }\\ 
{\small DateOfDeath }\\ 
{\small Titles}\\
{\small FounderOf}\\
{\small EmployeeOf}\end{tabular}
  &
   \begin{tabular}[b]{l}{\small Affiliate }\\ {\small Contact\_Meet\_Entity} \end{tabular} 
   & 
   \begin{tabular}{@{}l@{}}{\small Affiliate }\\ {\small TopMembers} \\ {\small FoundedBy}\end{tabular} \\ 
\hline 
\end{tabular} 
\end{table}
% Evaluation Criteria


% Quick Discussion of our approach

