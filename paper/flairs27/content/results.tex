
\section{Evaluation}
\label{sec:results}

%\ceg{We can add data on the total document size, The %number of each document type.  } 
 The purpose of evaluation is to measure the effectiveness of extracting slot values for various entities, metrics such as precision and recall would be considered. We perform random samplings to extract these. Our system was developed on a 32-core server described in Table~\ref{table:serverspec}. The corpus is a snapshot of the web in English. Each document is annotated using lingpipe and is called StreamItem, a bundle of StreamItems are put together and serialized as Apache Thrift objects, then compressed using xz compression with Lempel–Ziv–Markov chain algorithm (LZMA2) and finally encrypted using GNU Privacy Guard (GPG) with RSA asymmetric keys. The total size of the data after XZ compression and GPG encryption is 4.5TB and just over 500M StreamItems \cite{s3}. Data is stored in directories the naming of which is date-hour combination: from 2011-10-05-00 (5th of October 2011, 12am) until 2013-02-13-23 (13th of Feburary 2013, 11pm), which consists of 11952 date-hour combinations. The first four months of data (October 2011 - February 2012) is for training purposes, and we use this portion to lookup sample lines and for system parameter settings purposes. This corpora consists of various media types the distribution of which can be found in Table~\ref{table:documentsDist}. To have a sense of the scale of objects and compression as an example a 6mb gpg.xz files would become 45 mb thrift objects which can contain a couple of thousand StreamItems depending on their size. Some of the documents have null values for their annotation fields. The source code of our system is stored as an open source project where enthusiasts can also contribute to \cite{github}, also the relevant discussion mailing list is accessible here \cite{googlegroups}.
 
 
\begin{table}[b]
\caption{Document Chunk Distribution }
\centering
\label{table:documentsDist}

\begin{tabular}{|l|c|c|}
\hline 
\textbf{\# of Documents} & \textbf{Document Type} & \textbf{Slots Found}\\ 
\hline 
	Arxiv & 10988 & \\ \hline
 Classified & 34887 & \\ \hline
 Forum & 77674 & \\ \hline
 Linking & 12947 & \\ \hline
 Mainstream News & 141936 & \\ \hline
 Memetracker & 4137 & \\ \hline
 News & 280629 & \\ \hline
 Review & 6347 & \\ \hline
 	 Social & 688848 & \\ \hline
 Weblog & 740987 & \\ \hline

  
\hline 
\end{tabular} 
\end{table}


 
 
 
We have 172 extraction patterns covering each slot-name/entity-type combinations. Our final submission was named \textit{submission\_infer}. Our results are as follows: Document extraction using query entity matching with aliases, sentence extraction using alias matching and co-reference. Slot extraction using patterns, NER tags and NP tags. 158,052 documents with query entities, 17885 unique extracted slot values for 8 slots and 139 entities, 4 slots and 31 entities missing.

On the performance of our initial submission run we performed random sampling via two processes, the results of which are according to Table~\ref{table:initialresult}. You can view that we have had an accuracy of around 55\%, and about 15\% wrong entity identified and 30\% incorrect value extracted across all entities and slot types. Most of our issues for this submission were regarding poor slot value extraction patterns and incomplete aliases whih were tried to be mtigated later on. For our final submission, we provide a more detailed statistics, which has been elaborated in Table~\ref{table:finalresultrecall} and Table~\ref{table:finalresultaccuracy}. Table~\ref{table:finalresultrecall} shows the extent of search outreach for
each slot name. You can see that \textit{Affiliate} has been the slot name with highest hits and \textit{CauseOfDeath} our lowest hit with 0 instances found matching our patterns, after that \textit{AwardsWon} has been the next with 38 instances found. Affiliate is a very generic term and extracting real affiliates can be quite challenging using the extraction patterns provided. This can lead to noisy results. On the other hand for more precise terms our accuracy increases but we have less recall. Table~\ref{table:finalresultaccuracy} addresses the relative accuracy measure per slot value. There you can view that we have had the highest accuracy of 63.6\% for \textit{AssociateOf} and the lowest of 1\% - 5\%  for \textit{Affiliate}, \textit{Contact\_Meet\_PlaceTime} and \textit{EmployeeOf}.

\begin{table}
\caption{Benchmark Server Specifications }
\centering
\label{table:serverspec}
\begin{tabular}{| c | p{4.8cm} |}
\hline 
\textbf{Spec} & \textbf{Details} \\ \hline
Model & Dell xxx 32 cores \\ \hline 
OS & CentOS release 6.4 Final \\ \hline 
Software Stack & GCC version 4.4.7, Java 1.7.0\_25, Scala 2.9.2, SBT 0.12.3 \\ \hline 
 RAM & 64GB\\ \hline 
 Drives & 2x2.7TB disks, 6Gbps, 7200RPM\\ \hline 
\end{tabular} 
\end{table}







\begin{table}
\caption{Initial Performance Measure }
\centering
\label{table:initialresult}

\begin{tabular}{ | c | c | p{2cm} | p{13mm} |}
\hline 
 & \textbf{Correct} & \textbf{Incorrect Entity name} & \textbf{Incorrect Value} \\ 
\hline 
Sampling \#1 & 55\% & 17\% & 27\% \\ 
\hline Sampling \#2 & 54\% & 15\% & 31\%  \\ 
\hline 
\end{tabular} 
\end{table}


\begin{table}[h]




\caption{Recall Measure: Generic slot names like affiliate had the most recall, compared to less popular slot names e.g. DateOfDeath}
\centering
\label{table:finalresultrecall}
\begin{tabular}{|l|p{13mm}|p{22mm}|}
\hline 
 \textbf{Slot Name} & \textbf{Instances Found} & \textbf{Entity \hspace{5 mm} Coverage} \\ 
\hline 
Affiliate & 108598 & 80 \\ \hline 
AssociateOf & 25278 & 106 \\ \hline 
AwardsWon & 38 & 14 \\ \hline 
CauseOfDeath & 0 & 0 \\ \hline 
Contact\_Meet\_Entity & 191 & 8 \\ \hline 
Contact\_Meet\_PlaceTime & 5974 & 109 \\ \hline 
DateOfDeath & 87 & 14 \\ \hline 
EmployeeOf & 75 & 16 \\ \hline 
FoundedBy & 326 & 30 \\ \hline 
FounderOf & 302 &  29 \\ \hline 
Titles & 26823 & 118 \\ \hline 
TopMembers & 314 & 26 \\ \hline 

\end{tabular} 



\caption{Accuracy Measure: Accuracy of AffiliateOf was the best and Affiliate applied poorly due to ambiguity of being an affiliate of somebody/something}
\centering
\label{table:finalresultaccuracy}
\begin{tabular}{|l|p{10mm}|p{10mm}|p{11mm}|}
\hline 
 \textbf{Slot Name}  & \textbf{Correct} & \textbf{Wrong Entity} & {\small \textbf{Incorrect Value}} \\ 
\hline 
Affiliate & 1\% & 95\% & 5\% \\ \hline 
AssociateOf & 63.6\% & 9.1\% & 27.3\%  \\ \hline 
AwardsWon & 10\% & 10\% & 80\%  \\ \hline 
CauseOfDeath & 0\% & 0\% & 0\%  \\ \hline 
Contact\_Meet\_Entity & 21\% & 42\% & 37\%  \\ \hline 
Contact\_Meet\_PlaceTime & 5\% & 20\% & 85\%  \\ \hline 
DateOfDeath & 29.6\% & 71\% & 25\%  \\ \hline 
EmployeeOf & 5\% & 30\% & 65\%  \\ \hline 
FoundedBy & 62\% & 17\% & 21\%  \\ \hline 
FounderOf & 50\% & 0\% & 50\%  \\ \hline 
Titles & 55\% & 0\% & 45\%  \\ \hline 
TopMembers & 33\% & 17\% & 50\%  \\ \hline 

\end{tabular} 
\end{table}








% Note: Here is where we give as much stats as possible on our runs

