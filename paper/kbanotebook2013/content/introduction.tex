
\section{Introduction}

In this article we describe the approach the GatorDSR team from 
University of Florida has adopted in addressing the challenge at NIST’s Text 
REtrieval Conference (TREC) --- Knowledge Base Acceleration track 2013. There are two main sections designed for this track. Cumulative Citation Recommendation and Streaming Slot Filling. Due to the importance of knowledgebases, both of these tracks aim to accelerating populating them, hence the title Knowledgebase Acceleration (KBA).
Below we describe each of these tracks and their purposes in full detail.

An important challenge in maintaining Wikipedia.org (WP) which is the most popular free, web-based, collaborative, multilingual encyclopedia in the internet, is to make sure its contents are uptodate. Currently, there is considerable time lag between the publication date of cited news articles and the date of an edit to WP creating the citation. As an example, the median time lag for a sample of about 60,000 web pages cited by WP articles in the \textit{living\_people} category is over a year and the distribution has a long and heavy tail\cite{JFrank12}. It is also noted that the majority of Wikipedia entities have updates on their associated Wiki article much less frequently than their mention frequency. Such stale entries are the norm in any large knowledge base (KB), because the number of humans maintaining the knowledge base is far fewer than the number of entities in the KB. Further, the number of mentions is much larger than the number of entities\cite{JFrank12}. 


\subsection{Cumulative Citation Recommendation (CCR)}

For this track, assessors were instructed to ``use the wikipedia article to identify (disambiguate) the entity, and then imagine forgetting all info in the WP article and asking whether the text provides any information about the entity''\cite{JFrank12}. Documents would be divided according to two metrics: a) mentioning the entity or not, b) relevance level to the entity:
\begin{itemize}
\item \textbf{Mention} 
\begin{itemize}
\item \textbf{Mention:} Document explicitly mentions target entity, such as full name, partial name, nickname, pseudonym, title, stage name.
\item \textbf{Zero-mention:} Document does not directly mention target. Could still be relevant, e.g. metonymic references like ``this administration'' -- \textgreater ``Obama''. See also synecdoche. A document could also be relevant to target entity through relation to entities mentioned in document -- apply this test question: can I learn something from this document about target entity using whatever other information I have about entity?
\end{itemize}
\item \textbf{Relevance}
\begin{itemize}
\item \textbf{Garbage:} not relevant, e.g. spam.
\item \textbf{Neutral:} Not relevant, i.e. no info could be deduced about entity, e.g., entity name used in product name, or only pertains to community of target such that no information could be learned about entity, although you can see how an automatic algorithm might have thought it was relevant.
\item \textbf{Relevant:} Relates indirectly, e.g., tangential with substantive implications, or topics or events of likely
impact on entity.
\item \textbf{Central:} Relates directly to target such that you would cite it in the WP article for this entity, e.g. entity is a
central figure in topics/events.

\end{itemize}
\end{itemize}


\subsection{Streaming Slot Filling (SSF)}
The task is that given certain WP or Twitter entities 
(wiki/twitter URLs) and certain relations of interest (given in Table  \ref{table:slotNameOntology}), extract as many triple relations as possible (hence, slot filling). This can be used to automatically populate knowledgebases such as free-base or DBPedia 
or even fill-in the information boxes at Wikipedia. Below, you can view some examples of what it means to fill in a slot value; in each example there is a sentence of interest that we wish to extract slot values from, an entity that the slot value is related to, and a slot name which can be thought of as the topic of the slot value:

\noindent \textbf{Example 1:} ``Matthew DeLorenzo and Josiah Vega, both 14 years old and students 
at Elysian Charter School, were honored Friday morning by C-SPAN and received 
\$1,500 as well as an iPod Touch after winning a nationwide video contest.''

Entity:  http://en.wikipedia.org/wiki/Elysian\_Charter\_School

Slot name: Affiliate

Possible slot values: ``Matthew DeLorenzo'', ``Josiah Vega''

Incorrect slot values: ``C-SPAN'', ``iPod Touch''

\noindent \textbf{Example 2:} ``Veteran songwriters and performers Ben Mason, Jeff Severson and 
Jeff Smith will perform on Saturday, April 14 at 7:30 pm at Creative Cauldron 
at ArtSpace, 410 S. Maple Avenue.''

Entity: http://en.wikipedia.org/wiki/Jeff\_Severson

Slot name: Affiliate

Possible slot values: ``Ben Mason'', ``Jeff Severson'', ``Jeff Smith''

Incorrect slot values: ``Creative Caldron'', ``Art Space''

\noindent \textbf{Example 3:}  ``Lt. Gov. Drew Wrigley and Robert Wefald, a retired North Dakota 
district judge and former state attorney general, unveiled the crest Friday 
during a ceremony at the North Dakota Capitol.''

Entity: http://en.wikipedia.org/wiki/Hjemkomst\_Center

Slot name: Contact\_Meet\_PlaceTime

Slot value: ``Friday during a ceremony at the North Dakota Capitol''


\noindent In \textit{streaming} slot filling, we are only interested in new 
slot values that were not substantiated earlier in the stream corpus. In Table \ref{table:slotNameOntology} you can view some slot values 
and their types, the comprehensive description of which can be found at
\cite{tackbp} and \cite{aec}. The details of the metric for SSF will favor systems that most closely match the changes in the ground truth time line of slot values. This is done by searching for other documents that mentioned an entity and exactly matched the slot fill strings selected by the assessors.



In the rest of the paper, we address the following material. In Section 2, we sketch the main components of the system and their purposes. In Section 3, describe the details of how we designed and implemented certain components critical to the behavior of the system. Section 4 addresses the performance and the results we achieved. Section 5, goes through a discussion of the challenges we had to produce reliable results over the massive amount of information available, and how we overcome these issues. For similar approaches regarding last year's track you can refer to \cite{ji2011knowledge}.

%\ceg{Add a discussion of why machine driven kb population is important.}







\begin{table}[b]
\caption{Ontology of Slot Name Categories }
\centering
\label{table:slotNameOntology}

\begin{tabular}{|c|c|c|}
\hline 
\textbf{PER} & \textbf{FAC} & \textbf{ORG} \\ 
\hline 
\begin{tabular}{@{}l@{}}Affiliate \\ AssociateOf \\ Contact\_Meet\_PlaceTime \\ AwardsWon \\ 
DateOfDeath \\ 
CauseOfDeath\\
Titles\\
FounderOf\\
EmployeeOf\end{tabular}
  &
   \begin{tabular}[b]{l}Affiliate \\ Contact\_Meet\_Entity \end{tabular} 
   & 
   \begin{tabular}{@{}l@{}}Affiliate \\ TopMembers \\ FoundedBy\end{tabular} \\ 
\hline 
\end{tabular} 
\end{table}




 

%\ceg{How will each submission be evaluated?}

% Motivation


% KBA Task


% Evaluation Criteria


% Quick Discussion of our approach

